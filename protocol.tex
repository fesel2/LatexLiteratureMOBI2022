% simple template for protocols
% created by Lena Meßner
% last update: 14.06.2022

% !TeX spellcheck = en_GB
\documentclass[a4paper,12pt]{report}

\usepackage{lipsum}
\usepackage[T1]{fontenc}
\usepackage[utf8]{inputenc}
\usepackage[greek, ngerman, main=english]{babel}
\usepackage{microtype}
\usepackage{titlesec}
\usepackage{geometry}
\usepackage{setspace}
\usepackage{parskip}
\usepackage{natbib}
\usepackage{caption}
\usepackage{blindtext}
\usepackage{graphicx}
\usepackage{booktabs}
\usepackage{multirow}
\usepackage{amsmath}
\usepackage{amsfonts}
\usepackage{amssymb}
\usepackage[hyphens]{url}
\usepackage{listings}
\usepackage{placeins}
\usepackage{pdfpages}
\usepackage{icomma}
\usepackage{pdflscape}
\usepackage{gensymb}
\usepackage{cmll}
\usepackage{color}
\usepackage{xcolor}
\usepackage{acronym}
\usepackage{cell}
\usepackage{array}
\usepackage{varwidth}
\usepackage{subcaption}  % statt subfig

%----------------------------------------------------------------
%----        Relevant for this course                        ----
%----------------------------------------------------------------

\usepackage[backend=biber,
  isbn=false,                     % ISBN nicht anzeigen, gleiches geht mit nahezu allen anderen Feldern
  autocite=inline,                % regelt Aussehen für \autocite
                                  %      inline: Zitat in Klammern (\parancite)
                                  %      footnote: Zitat in Fußnoten (\footcite)
                                  %      plain: Zitat direkt ohne Klammern (\cite)
  style=ieee,         		        % Legt den Stil für die Zitate fest
                                  %      ieee: Zitate als Zahlen [1]
                                  %      alphabetic: Zitate als Kürzel und Jahr [Ein05]
                                  %      authoryear: Zitate Author und Jahr [Einstein (1905)]
  hyperref=true,                  % Hyperlinks für Zitate
]{biblatex}     

\addbibresource{literature.bib}


%----------------------------------------------------------------

%\newcommand{\changefont}[3]{\fontfamily{#1}\fontseries{#2}\fontshape{#3}\selectfont} 
%für andere Schriftart (unwichtig!)

\titleformat{\chapter}
[hang]
{\Large}
{\thechapter}
{10pt}
{\Large}
[{\titlerule[0.3pt]}]


\titlespacing*{\chapter}{0pt}{0pt}{15pt}[0pt] %{Einrücken}{Abstand oben}{Abstand unten}[Rand]


\titleformat{\section}
{\large}
{\thesection}
{5pt}
{\large}

\titlespacing*{\section}{0pt}{17pt}{8pt}[0pt]


\DeclareCaptionStyle{Captions}{labelfont={small, bf}, textfont={small}, aboveskip=0.3cm, belowskip=0.5cm}
\DeclareCaptionStyle{Ausnahme}{labelfont={small, bf}, textfont={small}, aboveskip=0.3cm, belowskip=0cm}

\captionsetup{style=Captions}      %captions sind die Bildbeschriftungen!!


\geometry{outer=25mm,
	inner=25mm,
	top=25mm,
	bottom=25mm}

\onehalfspacing

\renewcommand{\arraystretch}{1.1}  % definiert vertikalen Abstand der Tabellenzeilen
	

\begin{document}
\onehalfspacing
\pagenumbering{roman}
	
\begin{titlepage}
	\begin{flushleft}University of Heidelberg\\
	 Faculty of Biosciences\\
	 Molecular Biotechnology Bachelor Program\\
	\end{flushleft}
\vspace*{6.5cm}
	\begin{center}
		\huge Training state\\ \bigskip
    	\Large Seminar LATEX and Literature\\ \smallskip
    	\large 15.02.2022
    	\normalsize
	\end{center}
\vspace*{\fill}
	\begin{flushright} 
	\large Max Mustermann\\ \normalsize
	matriculation number 1234567
	\end{flushright}
\end{titlepage}

\tableofcontents
\addcontentsline{toc}{chapter}{Contents}

\newpage


\onehalfspacing
\pagenumbering{arabic}
\chapter{Introduction}
\blindtext



\chapter{Materials and Methods}
\lipsum[2-5]


\chapter{Results}

\lipsum[3]

\begin{equation*}
	f(x) = 0,264x - 0,0057
\end{equation*}

\begin{table}[h!]
	\centering
	\caption [Reaktionsraten]{In dieser Tabelle sind die berechneten Reaktionsraten mit der zugehörigen Konzentration an G6P dargestellt.\label{TabelleReaktionsraten}}
	
	\begin{tabular}{lp{1.5cm}l}
		\toprule
		c(G6P) [mM]	& &	Reaktionsrate [mM NADPH/s]	\\
		
		\midrule
		0	& &	(2,5 $\pm$ 0,7) $\cdot$ 10\textsuperscript{-5}	\\
		0,125	& &	(8,1 $\pm$ 0,3) $\cdot$ 10\textsuperscript{-5}	\\
		0,25	& &	(1,51 $\pm$ 0,18) $\cdot$ 10\textsuperscript{-4}	\\
		0,5	& &	(1,93 $\pm$ 0,12) $\cdot$ 10\textsuperscript{-4}	\\
		1	& &	(2,036 $\pm$ 0,003) $\cdot$ 10\textsuperscript{-4}	\\
		2	& &	(2,16 $\pm$ 0,06) $\cdot$ 10\textsuperscript{-4}	\\
		3	& &	(2,33 $\pm$ 0,07) $\cdot$ 10\textsuperscript{-4}	\\
		5	& &	(2,27 $\pm$ 0,09) $\cdot$ 10\textsuperscript{-4}	\\
		
		\bottomrule
	\end{tabular}
\end{table}


\FloatBarrier
		
\begin{align*}
f(x) &= 983,94x + 3796,8 = 0 \\
x &= -3,858771876
\end{align*}


\chapter{Discussion}

Eine andere vergleichbare Studie erhält ähnliche Ergebnisse und bestätigt damit die gestellte Hypothese \cite{douglas1900}. Es stimmt somit alles mit der Literatur überein. 

\appendix
\setcounter{chapter}{18}




%Literaturverzeichnis
\printbibliography



\end{document}